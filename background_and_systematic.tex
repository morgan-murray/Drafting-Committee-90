\section{Background Corrections and Systematic Uncertainties}
The extracted asymmetry amplitudes are subject to systematic uncertainties that
result from a combination of background processes, year-dependent
shifts in the missing-mass event distributions and various detector and binning
effects determined as in Ref\blue{s}.~\cite{Air08,Air09}. The
\blue{1996-2005} data sample also has a systematic uncertainty from misalignment of the spectrometer\blue{, which} was eliminated for the 2006-2007 data sample. Contributions to the uncertainties on the amplitude measurements arise from \blue{background in the data from} events that include neutral meson production. The procedure is described in detail in Refs.~\cite{Air08,Air09} and the contributions to the uncertainties of the measurements addressed here are given in Table~\ref{table_systematic_contributions_0607}. The largest \blue{contamination in} the data sample is expected to come from the \blue{excitation of a target proton to a $\Delta^+$ resonance. The separation of these events from the elastic sample is not feasible because the recoiling proton is not detected. No systematic uncertainty is assigned for the contribution from these events; they are treated as part of the signal}. A Monte Carlo calculation based on the parameterisation of Ref.~\cite{Bra76} is used to estimate the contribution to the event sample from resonant production in each kinematic bin. The results, denoted associated fraction and labelled ``assoc. fraction'', are shown in the last row of each figure in the results section. The method used to perform the estimation is described in detail in
Ref.~\cite{Air08}.

Another \blue{contribution to the} systematic uncertainty comes from the relative shift of the
missing-mass distribution\blue{s} between the electron and positron data
samples~\cite{Zei09,Bur10}. The exclusive region in the missing-mass
distribution is adjusted for each calorimeter calibration period for data
collected in 2007 to account for the observed time dependence. One
quarter of the difference between the asymmetries extracted using the standard
and shifted missing-mass windows is taken as the corresponding systematic
uncertainty. The beam polarisation measurement has a total uncertainty of 2.8\% and 3.4\% for the 1996-2005 and 2006-2007 data taking periods respectively,
which is present in the beam-helicity amplitudes and is quoted as an
independent scale uncertainty.

The predominant contribution to the systematic uncertainty arises from detector
effects, including the acceptance of the spectrometer, smearing effects \blue{due to detector resolution and radiative processes}, and
finite bin width in $-t$, $x_{\textrm{B}}$ and $Q^{2}$. In order to quantify
these effects, DVCS/BH events were generated using a Monte Carlo simulation of
the spectrometer that included these effects. The simulation used an event
generator based on the GPD model described in Ref.~\cite{Guz06} because it describes the data well and to keep consistency with Ref.~\cite{Air09}. Asymmetry amplitudes were extracted from these simulated events using the same analysis
procedure used to extract amplitudes from experimental data.  In each kinematic
bin, the systematic uncertainty was determined as the difference between the
asymmetry amplitude reconstructed from the simulated DVCS/BH data and that
calculated from the same GPD model at the average $-t$, $x_{\textrm{B}}$ and
$Q^{2}$ value for that bin.
The total systematic uncertainty for the 2006-2007 data sample was
determined for each kinematic bin from the uncertainty contributions of the
background corrections, detector effects and missing-mass shift
uncertainties added in quadrature. These contributions are \blue{summarised} in Table
\ref{table_systematic_contributions_0607}.

\begin{table}[H]
 \begin{center}
\resizebox{16cm}{!} {
 \begin{tabular}{|c|c|c|c|c|c|}
  \hline
 & Background & Missing-Mass Shift & Detector Effects & & Total \\
  \hline
  \hline
  $A_{\textrm{LU,I}}^{\sin\phi}$ & 0.005 & 0.005 & 0.025 & & 0.026 \\
  \hline
  $A_{\textrm{LU,I}}^{\sin(2\phi)}$ & 0.005 & 0.001 & 0.002 & & 0.006 \\
  \hline
  \hline
  $A_{\textrm{LU,DVCS}}^{\sin\phi}$  & 0.004 & 0.007 & 0.001 & & 0.008 \\
  \hline
  \hline
  $A_{\textrm{C}}^{\cos(0\phi)}$ & 0.001 & 0.004 & 0.007 & & 0.008 \\
  \hline
  $A_{\textrm{C}}^{\cos\phi}$ & 0.002 & 0.001 & 0.002 & & 0.003 \\
  \hline
  $A_{\textrm{C}}^{\cos(2\phi)}$ & 0.001 & 0.001 & 0.001 & & 0.002 \\
  \hline
  $A_{\textrm{C}}^{\cos(3\phi)}$ & 0.000 & 0.001 & 0.001 & & 0.001 \\
  \hline
 \end{tabular}
}
  \caption{The contributions to systematic uncertainty due to the
background correction, the time-dependence of the missing-mass
distributions and detector effects for the 2006-2007 data to the total
systematic uncertainty. The total systematic uncertainties, shown in the
right-most column, are the individual contributions added in quadrature.}
  \label{table_systematic_contributions_0607}
\end{center}
\end{table}
