\section{Results}
\blue{R}esults \blue{for} the beam-helicity and beam-charge asymmetry \blue{amplitudes} extracted separately from previously published 1996-2005~\cite{Air09} and the new 2006-2007 hydrogen data sets, are presented in Figs. \ref{release_bsa_0607} and \ref{release_bca_0607}. The asymmetry amplitudes extracted from the combined 1996-2007 hydrogen data set are shown in Figs.~\ref{bsa_xbjrange} and~\ref{bca_xbjrange}. Each of the asymmetry amplitudes is shown extracted in one bin over all kinematic variables (``overall'') and also projected against $-t$, $x_{\textrm{B}}$ and $Q^{2}$. All of the extracted asymmetry amplitudes \blue{are subject to} a \blue{fractional} contribution to the event yield from associated \blue{production (``Assoc. fraction''), which} is shown in the bottom row of each figure. The beam-helicity asymmetry amplitudes are subject to an additional scale uncertainty from the measurement of the beam polarisation, which is stated in the captions of the figures.
\begin{figure}
\begin{center}
\includegraphics[width=15cm,keepaspectratio]{bsadvcsplots_eml_par13_bin6_release_pic_update_0607_9605}
  \caption{Beam-helicity asymmetry amplitudes extracted separately from
the 1996-2005 (open triangles) and 2006-2007 (filled squares)
hydrogen data.   
An additional 2.8\,\% and 3.4\,\% scale uncertainty for the 1996-2005 and
2006-2007 data respectively is present in the amplitudes due to the \blue{im}precision of
the beam polarisation measurement. The inner error bars represent the statistical uncertainties\blue{, while} total error bars denote the statistical and systematic uncertainties added in quadrature. The \blue{simulated fractional} contribution \blue{from associated production} to the yield in each kinematic bin is shown in the bottom row.}
 \label{release_bsa_0607}
\end{center}
 \end{figure}
A \blue{statistical test was applied in order to check for possible incompatiability between the asymmetry amplitudes extracted from both data sets. Only the statistical uncertainties were employed in this test. The results} show that the \blue{the two} data sets are \blue{not incompatible}. The $A_{\textrm{LU,I}}$, $A_{\textrm{LU,DVCS}}$ and $A_{\textrm{C}}$ asymmetry amplitudes can therefore be extracted from the complete hydrogen data set recorded during the entire experimental operation of H{\sc ermes}.
\begin{figure}
\begin{center}
 \includegraphics[width=15cm,keepaspectratio]{bcaplots_eml_par13_bin6_release_pic_update_0607_9605_withassoc}
  \caption{Beam-charge asymmetry amplitudes extracted separately from the 1996-2005 (open triangles) and 2006-2007 (filled squares) hydrogen data.
The inner error bars represent the statistical uncertainties\blue{, while t}he total error bars denote the statistical and systematic uncertainties added in quadrature. The \blue{simulated fractional} contribution \blue{from associated production} to the yield in each kinematic bin is shown in the bottom row.}
 \label{release_bca_0607}
\end{center}
 \end{figure}

\begin{figure}
 \begin{center}
 \includegraphics[width=15cm]{bsadvcsplots_eml_par13_bin6_release_all_pic_update}
  \caption{The $A_{\textrm{LU,I}}^{\sin\phi}$, $A_{\textrm{LU,DVCS}}^{\sin\phi}$ and
$A_{\textrm{LU,I}}^{\sin(2\phi)}$ beam-helicity amplitudes extracted from all the hydrogen data recorded at H{\sc ermes}
from 1996 until 2007. The error bars (bands) represent the statistical
(systematic) uncertainties. Theoretical calculations from the model described in \cite{Kum09} are shown as solid and dashed lines. See text for details. The \blue{simulated fractional} contribution \blue{from associated production} to the yield in each kinematic bin is shown in the bottom row.}
  \label{bsa_xbjrange}
 \end{center}
\end{figure}

\begin{figure}
  \begin{center}
    \includegraphics[width=15cm]{bcaplots_eml_par13_bin6_all_release_pic_update_withassoc}
    \caption{The $A_{\textrm{C}}^{\cos(0\phi)}$, $A_{\textrm{C}}^{\cos\phi}$, $A_{\textrm{C}}^{\cos(2\phi)}$ and $A_{\textrm{C}}^{\cos(3\phi)}$ \blue{beam-charge} amplitudes extracted from all the hydrogen data recorded at H{\sc ermes} from 1996 until 2007. The error bars (bands) represent the statistical (systematic) uncertainties.  Theoretical calculations from the model described in \cite{Kum09} are shown as solid and dashed lines. See text for details. The \blue{simulated fractional} contribution \blue{from associated production} to the yield in each kinematic bin is shown in the bottom row.}
  \label{bca_xbjrange}
 \end{center}
\end{figure}
The results of the beam-helicity and beam-charge asymmetry amplitudes extracted from the complete 1996-2007 hydrogen sample are shown in Figs.~\ref{bsa_xbjrange} and \ref{bca_xbjrange}. The number of analysable DVCS/BH events available from the 2006-2007 data set \blue{(67815)} is approximately three times greater than the number events recorded in the 1996-2005 sample \blue{(24817)}. The  asymmetry amplitudes extracted from the complete 1996-2007 data set are thus weighted towards the 2006-2007 result. \blue{This is most noticeable in the set of beam-charge asymmetry amplitudes because the beam polarisation was lower in 2006 and 2007}.

The first and second harmonics of $\mathcal{A}_{\textrm{LU,I}}$, which are
sensitive to the interference term in the scattering amplitude, are shown in the first and third rows of Fig.~\ref{bsa_xbjrange}. The leading-twist amplitude $A_{\textrm{LU,I}}^{\sin\phi}$ has the largest value of any of the amplitudes when extracted in a single bin from the entire data set. This amplitude has no strong dependence on $-t$, $x_{\textrm{B}}$ or $Q^{2}$, implying a strong dependence at smaller values of $-t$ as the amplitude must diminish at low $t$ values due to the dependence of the amplitude on the factor $k$ from in Eq.~\ref{eq:s1}. The $A_{\textrm{LU,I}}^{\sin\phi}$ amplitude is sensitive to the imaginary part of the CFF $\mathcal{H}$ and thereby can constrain GPD $\textit{H}$. The twist-3 $A_{\textrm{LU,I}}^{\sin(2\phi)}$ amplitudes are compatible with zero, as are the twist-3 $A_{\textrm{LU,DVCS}}^{\sin\phi}$ amplitudes dependent on the squared DVCS term. These are shown in the second row of Fig.~\ref{bsa_xbjrange}. Neither of these amplitudes show any clear dependence on $-t$, $x_{\textrm{B}}$ or $Q^{2}$. The systematic error bands of the asymmetry amplitudes extracted from the combined data set were determined using Monte Carlo simulations that reflect the equipment used \blue{used in the various stages} of the H{\sc ermes} experiment. 

The $A_{\textrm{C}}^{\cos(n\phi)}$ amplitudes are shown in Fig.~\ref{bca_xbjrange}. The leading-twist $A_{\textrm{C}}^{\cos(0\phi)}$ and $A_{\textrm{C}}^{\cos\phi}$ amplitudes are both non-zero. There is a relationship between $A_{\textrm{C}}^{\cos(0\phi)}$ and $A_{\textrm{C}}^{\cos\phi}$ as the Fourier coefficient $c^{\textrm{I}}_{0,\textrm{UNP}}$ is inversely proportional to $c^{\textrm{I}}_{1,\textrm{UNP}}$ via the kinematic factor $k$ introduced in Eq~\ref{eq:c1}. These amplitudes diverge with opposite sign from zero at increasing values of $-t$ but they
have no discernible dependence on $x_{\textrm{B}}$ and $Q^{2}$. The $A_{\textrm{C}}^{\cos(2\phi)}$ and $A_{\textrm{C}}^{\cos(3\phi)}$ amplitudes are both consistent with zero and have no significant variation in value over the range in $-t$, $x_{\textrm{B}}$ and $Q^{2}$. The amplitude $A_{C}^{\cos(2\phi)}$ is related to twist-3 GPDs and $A_{\textrm{C}}^{\cos(3\phi)}$ relates to gluon helicity-flip GPDs. Both of these amplitudes are expected to be suppressed at H{\sc ermes} kinematic conditions compared to the leading twist amplitudes.

The curves in Figs.~\ref{bsa_xbjrange} and~\ref{bca_xbjrange} show calculations from a global fit of GPDs to experimental data\cite{Kum09}. The basic model is based on a minimalist dual representation of GPDs with only (very) weakly entangled skewedness and $t$ dependences. In the model, the $t$ dependence is approximated by a physically-motivated Regge dependence. The solid curves represent the model \blue{fit} without data from the experiment \cite{Gir08,Cam06} at Hall A in Jefferson Laboratory; the model \blue{fit} represented by the dashed curves incorporates this data in the fit for an extraction of the GPDs. Both fits include the 1996-2005 H{\sc ermes} data. The model incorporates only twist-2 GPDs and so can provide calculations only for the $A_{\textrm{LU,I}}^{\sin\phi}$, $A_{\textrm{C}}^{\cos(0\phi)}$ and $A_{\textrm{C}}^{\cos\phi}$ asymmetry amplitudes. The $A_{\textrm{LU,I}}^{\sin\phi}$ amplitude is well described by the model. \blue{Both the $A_{\textrm{C}}^{\cos(0\phi)}$ and $A_{\textrm{C}}^{\cos\phi}$ amplitudes favour the GPD fit that did not include the Hall A data.}

The asymmetry amplitudes shown in Figs.~\ref{release_bsa_0607} and \ref{release_bca_0607} were also re-binned as a function of $-t$ for three different ranges of $x_{\textrm{B}}$. There was no observed dependence of any of the amplitudes with this two dimensional binning.
