\section{Introduction}
Generalised Parton Distributions (GPDs)
\cite{Mue94,Ji97a,Rad97} encompass the familiar parton
distribution functions (PDFs) and nucleon form factors to provide a
comprehensive description of the structure of the nucleon.
Although GPDs are difficult to access experimentally, the H{\sc ermes}
collaboration has previously published results
\cite{Air01,Air06,Air08,Air09,Air10, Air10a, Air10b, Air11} on
the leptoproduction of real photons from a nucleon or nucleus that can be used to constrain GPDs via Compton Form Factors (CFFs). Each CFF is a convolution of a hard scattering kernel with a GPD that describes a soft part of the scattering process.

Generalised parton distributions depend upon four kinematic variables: the
Mandelstam variable $t=(p-p^{\prime})^2$, which is the squared momentum
transfer to the target nucleon in the scattering process with $p$ ($p^{\prime}$)
representing the initial (final) four-momentum of the proton; the average
fraction $x$ of the nucleon's longitudinal momentum carried by the active
quark throughout the scattering process; half the difference in the
change of the fraction of the nucleon's longitudinal momentum carried by the
active quark at the start and end of the process, written as the skewness
variable $\xi$; and the square $-Q^2=q^2$ of the four-momentum of the virtual photon that mediates the lepton-proton scattering process. In the Bjorken limit of $Q^2\rightarrow\infty$ with
fixed $t$, $\xi$ is related to the Bjorken variable
$x_{\textrm{B}}=\frac{-q^2}{2p\cdot q}$ as
$\xi\approx\frac{x_\textrm{B}}{2-x_\textrm{B}}$. The results are presented
against $x_{\textrm{B}}$ because there is no consensus on an experimentally
observable representation of $\xi$. 

\begin{figure}
\begin{center}
\subfigure[DVCS]{\includegraphics[width=5.6cm]{dvcs}}
\hspace{3cm}
\subfigure[Bethe Heitler]{\includegraphics[width=5.2cm]{1_bh}}
\caption[DVCS and Bethe Heitler hand bag diagram.]{(a): The DVCS process in
which an electron/positron ($e$) interacts with a quark in the nucleon ($p$) via a virtual photon ($\gamma^\ast$). The quark is found in the nucleon with longitudinal momentum fraction $x+\xi$ and emits a real photon ($\gamma$). The quark is absorbed by the nucleon with fraction $x-\xi$. (b): The Bethe-Heitler process which has the same initial and final states as DVCS.}
\label{spin}
\end{center}
\end{figure}

Leptoproduction of real photons ($e\,p\,\rightarrow\,e'\,p'\,\gamma$) can be described in terms of the two experimentally indistinguishable processes shown in Fig.~\ref{spin}: the Deeply Virtual Compton Scattering (DVCS) process, which is the
emission of a real photon by a quark from the nucleon, and the Bethe-Heitler (BH) process, which is elastic lepton-proton scattering with the
emission of a Bremsstrahlung photon by the lepton. 
The BH process is calculable in the QED framework; this process is
dominant at the kinematic conditions of the H{\sc ermes} experiment, but the
scattering amplitudes of the two processes interfere and the large BH amplitude
amplifies the contribution of the DVCS amplitude to the interference term. 
It is through the study of this interference term at H{\sc ermes} that
useful information for the constraint of certain GPDs can be obtained \cite{Bel02b}.

The differential four-fold cross section for the leptoproduction of real photons
from an unpolarised target can be written \cite{Bel02b}
\begin{center}
\begin{equation}
\frac{\textrm{d}^4\sigma}{\textrm{d}x_{\textrm{B}}\textrm{d}Q^{2}\textrm{d}
|t|\textrm{d}\phi} =
\frac{x_{\textrm{B}}e^{6}}{32(2\pi)^{4} Q^{4}\sqrt{1+\epsilon^{2}}}
|\tau|^{2},
\end{equation}
\end{center}
where $e$ is the elementary
charge, $\epsilon=2x_\textrm{B}\frac{M}{Q}$ with $M$
the nucleon mass, and $\phi$ is the
azimuthal angle between the scattering and production planes \cite{Tre04}.
The scattering amplitude $\tau$ can be written
\begin{center}
\begin{equation}
|\tau|^{2} = |\tau_{\textrm{BH}}|^{2} +
|\tau_{\textrm{DVCS}}|^{2} + \textrm{I},
\end{equation}
\end{center}
with contributions from the \textrm{BH} process ($\tau_{\textrm{\textrm{BH}}}$),
the DVCS process
($\tau_{\textrm{DVCS}}$) and the interference term (I). These
contributions can be written 
\begin{equation}
 |\tau_{\textrm{BH}}|^{2} =
\frac{K_{\textrm{BH}}}{\mathcal{P}_{1}(\phi)\mathcal{P}_{2}(\phi)}
\left(c_{0}^{
\textrm{BH}} + \sum_{n=1}^2 c_{n}^{\textrm{BH}}\cos(n\phi)\right),
\label{e:tbh}
\end{equation}
\begin{equation}
 \hspace{2.2cm}|\tau_{\textrm{DVCS}}|^{2} =
K_{\textrm{DVCS}}\left(c_{0}^{\textrm{DVCS}} +
\sum_{n=1}^2
c_{n}^{\textrm{DVCS}}\cos(n\phi) + \lambda
s_{1}^{\textrm{DVCS}}\sin\phi\right),
\label{e:tdvcs}
\end{equation}
\begin{equation}
\hspace{3.7cm} \textrm{I} = \frac{- e_\ell
K_{\textrm{I}}}{\mathcal{P}_{1}(\phi)\mathcal{P}_{2}(\phi)}\left(c_{0}^{\textrm{
I}}
+
\sum_{n=1}^3 c_{n}^{\textrm{I}}\cos(n\phi) + \lambda \sum_{n=1}^2
s_{n}^{\textrm{I}}\sin(n\phi)\right),
\label{e:ti}
\end{equation}
where $\mathcal{P}_1(\phi)$ and $\mathcal{P}_2(\phi)$ are the lepton propagators
of the BH process and $\lambda$ and $e_\ell$ are respectively the
helicity and unit charge of the lepton beam.  The
quantities $K_{\textrm{BH}}=1/(x_\textrm{B}^2t(1+\epsilon^2)^2)$,
$K_{\textrm{DVCS}}=1/Q^2$
and $K_{\textrm{I}}=1/(x_{\textrm{B}}yt)$ are kinematic factors, where
$y$ is the fraction of the beam energy carried by the virtual photon in
the target rest frame. A full
explanation of the Fourier coefficients [$c_n^V,s_n^W$]
($V\in[\textrm{BH,DVCS},\textrm{I}]$ and $W\in[\textrm{DVCS},\textrm{I}]$) can be
found in
Ref.~\cite{Bel02b}.
 
There are two sets of asymmetries that are measured at H{\sc
ermes} with an unpolarised target and a polarised electron or positron beam:
beam-helicity asymmetries and beam-charge asymmetries. In order to disentangle the DVCS and interference contributions
to the cross section via the beam-charge dependence of the
interference term this paper, as in Ref.~\cite{Air09}, presents results on the following asymmetries:
\begin{align}
\hspace{0.5cm}\mathcal{A}^{\textrm{I}}_{\textrm{LU}}(\phi) &\equiv
\frac{(\textrm{d}\sigma(\phi)^{+\rightarrow} -
\textrm{d}\sigma(\phi)^{+\leftarrow}) -
(\textrm{d}\sigma(\phi)^{-\rightarrow}
- \textrm{d}\sigma(\phi)^{-\leftarrow})}{(\textrm{d}\sigma(\phi)^{+\rightarrow}
+
\textrm{d}\sigma(\phi)^{+\leftarrow}) +
(\textrm{d}\sigma(\phi)^{-\rightarrow}
+ \textrm{d}\sigma(\phi)^{-\leftarrow})}&  \nonumber \\
&=\dfrac{-\dfrac{K_{\textrm{I}}}{\mathcal{P}_{1}(\phi)\mathcal{P}_{2}(\phi)}
\displaystyle\sum_{n=1}^2
s_{n}^{\textrm{I}}\sin(n\phi)}{\dfrac{K_{\textrm{BH}}}{\mathcal{P}_{1}
(\phi)\mathcal{P}_{
2}(\phi)}
\displaystyle\sum_{n=0}^2
c_{n}^{\textrm{BH}}\cos(n\phi) + 
K_{\textrm{DVCS}}\displaystyle\sum_{n=0}^2 c_{n}^{\textrm{DVCS}}\cos(n\phi)},& 
\label{e:alui}
\end{align}

\begin{align}
\mathcal{A}^{\textrm{DVCS}}_{\textrm{LU}}(\phi) &\equiv
\frac{(\textrm{d}\sigma(\phi)^{+\rightarrow} +
\textrm{d}\sigma(\phi)^{-\rightarrow}) -
(\textrm{d}\sigma(\phi)^{+\leftarrow} + 
\textrm{d}\sigma(\phi)^{-\leftarrow})}
{(\textrm{d}\sigma(\phi)^{+\rightarrow} +
\textrm{d}\sigma(\phi)^{-\rightarrow}) +
(\textrm{d}\sigma(\phi)^{+\leftarrow}
+ \textrm{d}\sigma(\phi)^{-\leftarrow})}&  \nonumber \\
&=\dfrac{K_{\textrm{DVCS}}
s_{1}^{\textrm{DVCS}}\sin\phi}{\dfrac{K_{\textrm{BH}}}{\mathcal{P}_{1}
(\phi)\mathcal{P}_{2}(\phi)}
\displaystyle\sum_{n=0}^2
c_{n}^{\textrm{BH}}\cos(n\phi) + 
K_{\textrm{DVCS}}\displaystyle\sum_{n=0}^2
c_{n}^{\textrm{DVCS}}\cos(n\phi)} \textrm{\,\,and}&
\label{e:aludvcs}
\end{align}

\begin{align}
\hspace{0.5cm}\mathcal{A}_{\textrm{C}}(\phi) &\equiv  
\frac{(\textrm{d}\sigma(\phi)^{+\rightarrow} +
\textrm{d}\sigma(\phi)^{+\leftarrow}) -
(\textrm{d}\sigma(\phi)^{-\rightarrow}
+ \textrm{d}\sigma(\phi)^{-\leftarrow})}{(\textrm{d}\sigma(\phi)^{+\rightarrow}
+
\textrm{d}\sigma(\phi)^{+\leftarrow}) +
(\textrm{d}\sigma(\phi)^{-\rightarrow}
+ \textrm{d}\sigma(\phi)^{-\leftarrow})}&    \nonumber \\
&=\dfrac{{-\dfrac{K_{\textrm{I}}}{\mathcal{P}_{1}(\phi)\mathcal{P}_{2}(\phi)}
\displaystyle\sum_{n=0}^3
c_{n}^{\textrm{I}}\cos(n\phi)}}{\dfrac{K_{\textrm{BH}}}{\mathcal{P}_{1}
(\phi)\mathcal{P}_
{2}(\phi)}
\displaystyle\sum_{n=0}^2
c_{n}^{\textrm{BH}}\cos(n\phi) + 
K_{\textrm{DVCS}}\displaystyle\sum_{n=0}^2 c_{n}^{\textrm{DVCS}}\cos(n\phi)} ,&
\label{e:ac}
\end{align}
where $\textrm{d}\sigma(\phi)^+$ ($\textrm{d}\sigma(\phi)^-$) refers to
cross sections with positive (negative) beam charge and
$\textrm{d}\sigma(\phi)^\rightarrow$ ($\textrm{d}\sigma(\phi)^\leftarrow$) refer
to cross sections taken with beam spin parallel (anti-parallel) to the
beam momentum.

The analysis presented here additionally includes a larger,
independent data set taken during the years 2006 and 2007 after the
experiment was upgraded in the target region. The results in this paper do not make use of any of the information from the
upgraded detector equipment (i.e. a missing-mass technique for event selection is used), but represent the largest DVCS data set that will
be published by H{\sc ermes}.
