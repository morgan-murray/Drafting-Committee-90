\section{Summary}

Beam\blue{-}helicity and beam\blue{-}charge asymmetries in the azimuthal distribution of real photons from leptoproduction on an unpolarised hydrogen target have been presented. These asymmetries were extracted from a new unpolarised hydrogen data set taken during the 2006 and 2007 operating period of H{\sc ermes}. \blue{An analogous set of asymmetry amplitudes were extracted previously from hydrogen data obtained during the 1996-2005 experimental period as described in Ref.~\cite{Air09}}. A comparison of the amplitudes extracted from these independent data sets have shown that they are compatible and \blue{the }asymmetry amplitudes can therefore be extracted from the complete 1996-2007 event sample. The resulting beam\blue{-}charge asymmetry amplitudes represent the most statistically precise DVCS result that will be published by the H{\sc ermes} collaboration. The result of this extraction of asymmetry amplitudes show that there is a strong signal in the first harmonic of the interference contribution to the beam\blue{-}helicity asymmetry. There are also non zero amplitudes in the zeroth and first harmonic\blue{s} of the beam\blue{-}charge asymmetry. Each of the other higher order asymmetry amplitudes are consistent with zero. The combined results are compared to calculations from ongoing work to fit GPD models to experimental data. \blue{One} model fit~\cite{Kum09} includes the H{\sc ermes} data set from 1996-2005 \blue{shown in} Ref.~\cite{Air09}. The model achieves good agreement with the presented beam-helicity asymmetry amplitude results, and fair agreement with the beam-charge asymmetry amplitudes. \blue{The other model fit~\cite{Liu11} does not include the H{\sc ermes} data from 1996-2005 and fails to describe the presented data as accurately. A third set of calculations is presented that arise from efforts to extract the Compton Form Factors from experimental data at discrete kinematic points~\cite{Gui11}. The fit on which these calculations are based includes the H{\sc ermes} data set from 1996-2005. The calculations describe the data well, but cannot be extrapolated past the H{\sc ermes} kinematic domain.} All \blue{data} amplitudes are also presented \blue{as projections} in $-t$ in bins of $x_{\textrm{B}}$. No localised features are observed in any particular $t$-bin.

\acknowledgments
We gratefully acknowledge the \desy\ management for its support and the staff
at \desy\ and the collaborating institutions for their significant effort.
This work was supported by 
the Ministry of Economy and the Ministry of Education and Science of Armenia;
the FWO-Flanders and IWT, Belgium;
the Natural Sciences and Engineering Research Council of Canada;
the National Natural Science Foundation of China;
the Alexander von Humboldt Stiftung,
the German Bundesministerium f\"ur Bildung und Forschung (BMBF), and
the Deutsche Forschungsgemeinschaft (DFG);
the Italian Istituto Nazionale di Fisica Nucleare (INFN);
the MEXT, JSPS, and G-COE of Japan;
the Dutch Foundation for Fundamenteel Onderzoek der Materie (FOM);
the Russian Academy of Science and the Russian Federal Agency for 
Science and Innovations;
the U.K.~Engineering and Physical Sciences Research Council, 
the Science and Technology Facilities Council,
and the Scottish Universities Physics Alliance;
the U.S.~Department of Energy (DOE) and the National Science Foundation (NSF);
and the European Community Research Infrastructure Integrating Activity
under the FP7 "Study of strongly interacting matter (HadronPhysics2, Grant
Agreement number 227431)".

