\section{Experiment and Data Selection}
The new data presented in this work were collected in 2006 and 2007. As in ref.~\cite{Air09}, the data were collected with the H{\sc ermes}
spectrometer \cite{Ack98} using the longitudinally polarised 27.6\,GeV
electron and positron beams incident upon an unpolarised hydrogen gas
target internal to the H{\sc era} lepton storage ring at D{\sc esy}. The integrated luminosities of the electron and positron data samples, respectively, are
approximately 246\,pb$^{-1}$ and 1460\,pb$^{-1}$, with average beam polarisations of 30.3\,\% and 39.2\% \cite{Ben01}. The procedure used to select events is similar to that used in ref.~\cite{Air09}.
A brief summary of this procedure is outlined in the following; consult
refs.~\cite{Zei09,Bur10} for more details. 

An event
was selected as having exactly one photon and one lepton
track detected within the acceptance of the spectrometer.
The event selection is subject to the kinematic constraints 1\,GeV$^{2}$ $<$
Q$^{2}$ $<$ 10\,GeV$^{2}$, 0.03 $<$ $x_{\textrm{B}}$ $<$ 0.35,
$-t < 0.7$\,GeV$^2$, $W^{2}$ $>$
9\,GeV$^{2}$ and $\nu$ $<$ 22\,GeV, where $W$ is the invariant mass of the
$\gamma^{*}p$ system and $\nu$ is the energy of the virtual photon in the target
rest frame. The polar angle between the virtual and real photons was required to
be within the limits 5 mrad $<$
$\theta_{\gamma^{*}\gamma}$ $<$ 45 mrad. 

An event sample was selected using a missing-mass technique requiring
that the squared missing-mass $M_{\textrm{X}}^{2}=(q+P-q')^{2}$
of the $e\,p \rightarrow\, e'\,\gamma\, \textrm{X}$ measurement
corresponded to the square of the proton mass within the limits of the
energy resolution of the H{\sc ermes} calorimeter. Here, $q$ is the
four-momentum of the virtual photon, $p$ is the initial four-momentum
of the target proton and $q'$ is the four-momentum of the produced
photon. The ``exclusive region'' was defined as $-$($1.5$\,GeV)$^{2} <
M_{\textrm{X}}^{2}$ $<$ (1.7\,GeV)$^{2}$, as in
ref.~\cite{Air09}. This exclusive region was shifted by up to
0.17\,GeV$^{2}$ \red{for certain subsets of the data in order }to reflect observed differences in the distributions of the electron and positron data samples~\cite{Bur10}. 

\section{Experimental Extraction of Asymmetry Amplitudes}

The expectation value of the experimental yield $N$ is parameterised as
\begin{equation}
 \langle N(e_{\ell},P_{\ell},\phi)\rangle =
\mathcal{L}(e_{\ell},P_{\ell})\eta(e_{\ell},\phi)\textrm{d}\sigma_{\textrm{UU}}
(\phi)\times
[1+P_{\ell}\mathcal{A}_{\textrm{LU}}^{\textrm{DVCS}}(\phi)+e_{\ell}P_{\ell}
\mathcal{A}_{\textrm{LU}}^{\textrm{I}}(\phi)+e_{\ell}\mathcal{A}_{\textrm{C}}
(\phi)],
\end{equation}
where $P_\ell$ is the beam polarisation, $\mathcal{L}$ is the integrated luminosity, $\eta$ is the detection
efficiency and d$\sigma_{\textrm{UU}}$ denotes the
cross section for an unpolarised target integrated over both beam charges and
beam helicities. The asymmetries $\mathcal{A}_{\textrm{LU}}^{\textrm{I}}(\phi)$,
$\mathcal{A}_{\textrm{LU}}^{\textrm{DVCS}}(\phi)$ and
$\mathcal{A}_{\textrm{C}}(\phi)$ are expanded in
$\phi$ as
\begin{equation}
\mathcal{A}_{\textrm{LU}}^{\textrm{I}}(\phi) \simeq \sum^{2}_{n=1}
A_{\textrm{LU,I}}^{\sin(n\phi)}\sin(n\phi) 
+ \sum^{1}_{n=0} A_{\textrm{LU,I}}^{\cos(n\phi)}\cos(n\phi), 
\label{alui_asym}
\end{equation}
\begin{equation}
 \mathcal{A}_{\textrm{LU}}^{\textrm{DVCS}}(\phi) \simeq 
A_{\textrm{LU,DVCS}}^{\sin\phi}\sin\phi 
+ \sum^{1}_{n=0} A_{\textrm{LU,DVCS}}^{\cos(n\phi)}\cos(n\phi), 
\label{aludvcs_asym}
\end{equation}
\begin{equation}
\mathcal{A}_{\textrm{C}}(\phi) \simeq \sum^{3}_{n=0}
A_{\textrm{C}}^{\cos(n\phi)}\cos(n\phi) 
+ A_{\textrm{C}}^{\sin\phi}\sin\phi,
\label{ac_asym}
\end{equation}
where the approximation is due to the truncation of the infinite
Fourier series that would describe the fitted distribution
  perfectly. Only the $\sin(n\phi)$ terms of the
$\mathcal{A}_{\textrm{LU}}$ asymmetries and the $\cos(n\phi)$ terms of the
$\mathcal{A}_{\textrm{C}}$ asymmetry are physically motivated. The other terms
are included as both a consistency check for any off-phase
extraneous harmonics in the data and as a test of the
normalisation of the fit. These terms are expected to be
consistent with zero and are found to be so.

A maximum-likelihood fitting technique \cite{Bar90} was used to
extract the asymmetry amplitudes in each kinematic bin of $-t$, $x_{\textrm{B}}$ and $Q^{2}$.
This method, described in ref.~\cite{Air08}, fits the expected
azimuthal distribution function to the data without introducing binning effects in $\phi$.
Event weights are introduced in the fitting procedure to account for
luminosity imbalances with respect to the beam charge and polarisation.

The asymmetry amplitudes $A_{\textrm{LU,I/DVCS}}^{\sin(n\phi)}$ and
$A_{\textrm{C}}^{\cos(n\phi)}$ relate respectively to the Fourier
coefficients $s_{n,\textrm{unp}}^{W}$ and $c_{n,\textrm{unp}}^{I}$ from the interference and
DVCS terms in Eqs.~\ref{e:alui}-\ref{e:ac}. The asymmetry amplitudes
may also be affected by the lepton propagators and the other
$\phi$-dependent terms in the denominators in
Eqs.~\ref{e:alui}-\ref{e:ac}. The  $s_{n,\textrm{unp}}^{W}$ and $c_{n,\textrm{unp}}^{V}$ Fourier
coefficients depend on ``$\mathcal{C}$-functions'' \cite{{Bel02b}}, each of which is a
combination of complex Compton Form Factors (CFF\red{s}). A CFF is a convolution of a GPD with a hard scattering kernel. Contributions of GPDs to the cross section may be relatively suppressed by powers of $1/Q$, the origin of which may be kinematic in nature or due to the twist level of the GPD. Leading twist is twist-2. Typically, the contribution of a twist-$n$ GPD, and hence the corresponding CFF, is suppressed by $\mathcal{O}(1/Q^{n-2})$. Table~\ref{tab_amplitudes} presents the asymmetry amplitudes extracted in this analysis and the related Fourier coefficients, dominant $\mathcal{C}$-function, twist-level and level of suppression.

At H{\sc ermes} kinematic conditions, the leading-twist beam-helicity asymmetry amplitude is $A_{\textrm{LU,I}}^{\sin\phi}$ and the leading beam-charge asymmetry amplitude is $A^{\cos\phi}_{\textrm{C}}$. These asymmetry amplitudes, for an unpolarised target, are sensitive respectively to Fourier coefficients $s_{1,\textrm{unp}}^{\textrm{I}}$ and $c_{1,\textrm{unp}}^{\textrm{I}}$. Both of these Fourier coefficients have a dominant contribution from the $\mathcal{C}_{\textrm{unp}}^{\textrm{I}}$-function:
\begin{eqnarray}
s_{1,\textrm{unp}}^{\textrm{I}} \approx&\,\,\,\,8\lambda ky(2-y)\mathfrak{Im}\mathcal{C}_{\textrm{unp}}^{\textrm{I}}\quad\red{\textrm{and}}&\label{eq:s1}
\\
 c_{1,\textrm{unp}}^{\textrm{I}} \approx&8k(2- 2y + y^{2})\mathfrak{Re}\mathcal{C}_{\textrm{unp}}^{\textrm{I}}\,,&\label{eq:c1}
\end{eqnarray}
where the definition of the kinematic factor $k$ is found
  in ref.\cite{Bel02b}. The $\mathcal{C}_{\textrm{unp}}^{\textrm{I}}$-function can be
written
\cite{Bel02b} 
\begin{equation}
 \mathcal{C}_{\textrm{unp}}^{\textrm{I}} = F_{1}\mathcal{H} + \frac{x_{\textrm{B}}}{2-x_{\textrm{B}}}(F_{1}+F_{2})\widetilde{\mathcal{H}} -\frac{t}{4M^{2}}F_{2}\mathcal{E},
\label{Eq_cunp}
\end{equation}
where $F_{1}$ and $F_{2}$ are respectively the Dirac and Pauli form
factors of the nucleon and $\mathcal{H}$, $\widetilde{\mathcal{H}}$ and
$\mathcal{E}$ are CFFs that relate respectively to the GPDs $H$,
$\widetilde{H}$ and $E$.  At H{\sc ermes} kinematic
conditions (where $x_{\textrm{B}}$ and $\frac{-t}{4M^2}$ are of order 0.1), the
contributions of CFFs $\widetilde{\mathcal{H}}$ and $\mathcal{E}$ can be
neglected in Eq.~\ref{Eq_cunp} (in first approximation) with respect to $\mathcal{H}$ since they
are kinematically suppressed by an order of magnitude or more.
Hence, the behaviour of
$\mathcal{C}_{\textrm{unp}}^{\textrm{I}}$ is determined by CFF $\mathcal{H}$
and therefore GPD $H$ can be constrained through
measurements of $A_{\textrm{LU,I}}^{\sin\phi}$ and $A^{\cos\phi}_{\textrm{C}}$.
The asymmetry amplitudes $A_{\textrm{LU},\textrm{I}}^{\sin(2\phi)}$,
$A^{\cos(0\phi)}_{\textrm{C}}$ and $A^{\cos(2\phi)}_{\textrm{C}}$ also depend
upon this $\mathcal{C}$-function. However, the $A^{\cos(0\phi)}_{\textrm{C}}$ amplitude is kinematically suppressed compared to the $A^{\cos\phi}_{\textrm{C}}$ amplitude, and the  $A_{\textrm{LU,I}}^{\sin(2\phi)}$ and $A^{\cos(2\phi)}_{\textrm{C}}$ amplitudes are higher\red{-}twist quantities (see Table~\ref{tab_amplitudes}).

The DVCS asymmetry amplitude $A^{\sin\phi}_{\textrm{LU,DVCS}}$ has a
contribution from the $\mathcal{C}_{\textrm{unp}}^{\textrm{DVCS}}$\red{-}fnction,
which is bilinear in CFFs. However, this twist-3 amplitude is inherently small at HERMES kinematic conditions due to the size of the $s_{1}^{\textrm{DVCS}}$ Fourier coefficient compared to the contribution from the $c_{n}^{\textrm{BH}}$ coefficients in the denominator of Eq.~\ref{e:aludvcs}. As a result of
the more complicated dependence on the CFFs and this suppression, it
is more difficult to constrain GPDs
via the measurement of $A^{\sin\phi}_{\textrm{LU,DVCS}}$ than from the
kinematically-unsuppressed leading-twist amplitudes.

The $A^{\cos(3\phi)}_{\textrm{C}}$ amplitude depends on the
$c_{3,\textrm{unp}}^{\textrm{I}}$ Fourier coefficient and hence the
$\mathcal{C}_{\textrm{T,unp}}^{\textrm{I}}$\red{-}ffunction. Although the CFFs
in this function are of leading twist, they relate to gluon helicity-flip GPDs and are thus suppressed by $\alpha_{\textrm{S}}/\pi$,
where $\alpha_{\textrm{S}}$ is the strong coupling constant.
\begin{table}
\begin{center}
\resizebox{\textwidth}{!} {
\begin{tabular}{|c|c|c|c|c|}
\hline
Asymmetry Amplitude& Fourier Coefficient& Dominant CFF Dependence & Twist Level & Relative Suppression   \\ 
\hline
\hline
$A_{\textrm{LU,I}}^{\sin\phi}$ & $s_{1,\textrm{unp}}^{\textrm{I}}$  &
$\mathfrak{Im}\mathcal{C}_{\textrm{unp}}^{\textrm{I}}$
&  2 & --\\
\hline
$A_{\textrm{LU,I}}^{\sin(2\phi)}$ & $s_{2,\textrm{unp}}^{\textrm{I}}$ 
&
$\mathfrak{Im}\mathcal{C}_{\textrm{unp}}^{\textrm{I}}$
&  3 & $1/Q$\\
\hline
\hline
$A_{\textrm{LU,DVCS}}^{\sin\phi}$ & $s_{1, \textrm{unp}}^{\textrm{DVCS}}$ &
$\mathfrak{Im}\mathcal{C}_{\textrm{unp}}^{\textrm{DVCS}}$ &  3 & $1/Q$ \\
\hline
\hline
$A_{\textrm{C}}^{\cos(0\phi)}$ & $c_{0,\textrm{unp}}^{\textrm{I}}$  &
$\mathfrak{Re}\mathcal{C}_{\textrm{unp}}^{\textrm{I}}$ & 2& $\frac{(2-y)^2K}{(1-y)(2-2y+y^2)}$
\\
\hline
$A_{\textrm{C}}^{\cos\phi}$ & $c_{1,\textrm{unp}}^{\textrm{I}}$  &
$\mathfrak{Re}\mathcal{C}_{\textrm{unp}}^{\textrm{I}}$ & 2 & --
\\
\hline
$A_{\textrm{C}}^{\cos(2\phi)}$ & $c_{2,\textrm{unp}}^{\textrm{I}}$ &
$\mathfrak{Re}\mathcal{C}_{\textrm{unp}}^{\textrm{I}}$ & 3 & $1/Q$ \\
\hline
$A_{\textrm{C}}^{\cos(3\phi)}$ & $c_{3,\textrm{unp}}^{\textrm{I}}$ &
$\mathfrak{Re}\mathcal{C}_{\textrm{T,unp}}^{\textrm{I}}$ &  2 & $\alpha_{\textrm{S}}/\pi$ \\
\hline
 \end{tabular}
}
\caption{Asymmetry amplitudes that can
be extracted from the available data set, the related Fourier
coefficients, dominant $\mathcal{C}$-function, twist-level and level of
suppression relative to the leading amplitudes at H{\sc ermes} kinematic conditions. {The constant K is defined in ref.~\cite{Bel02b}}.}
\label{tab_amplitudes}
\end{center}
\end{table}
 
