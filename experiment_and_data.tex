\section{Experiment and Data Selection}
The new data presented in this work were collected in 2006 and 2007. As in ref.~\cite{Air09}, the data were collected with the H{\sc ermes}
spectrometer \cite{Ack98} using the longitudinally polarised 27.6\,GeV
electron and positron beams incident upon an unpolarised hydrogen gas
target internal to the H{\sc era} lepton storage ring at D{\sc esy}. The integrated luminosities of the electron and positron data samples are
approximately 246\,pb$^{-1}$ and 1460\,pb$^{-1}$, with average beam polarisations of $0.303$ and $0.392$ respectively. The procedure used to select events is similar to that used in ref.~\cite{Air09}. A brief summary of this procedure is outlined in the following; more details are given in refs.~\cite{Zei09,Bur10}.

Events were selected if having exactly one photon and one lepton
track detected within the acceptance of the spectrometer.
The event selection is subject to the kinematic constraints 1\,GeV$^{2}$ $<$
Q$^{2}$ $<$ 10\,GeV$^{2}$, 0.03 $<$ $x_{\textrm{B}}$ $<$ 0.35,
$-t < 0.7$\,GeV$^2$, $W^{2}$ $>$
9\,GeV$^{2}$ and $\nu$ $<$ 22\,GeV, where $W$ is the invariant mass of the
$\gamma^{*}p$ system and $\nu$ is the energy of the virtual photon in the target
rest frame. The polar angle between the directions of the virtual and real photons was required to
be within the limits 5~mrad $<$
$\theta_{\gamma^{*}\gamma}$ $<$ 45~mrad. 

An event sample was selected requiring
that the squared missing-mass $M_{\textrm{X}}^{2}=(q+M_{p}-q')^{2}$
of the $e\,p \rightarrow\, e'\,\gamma\, \textrm{X}$ measurement
corresponded to the square of the proton mass, $M_{p}$, within the limits of the
energy resolution of the H{\sc ermes} spectrometer (mainly the calorimeter). Recall that $q$ is the
four-momentum of the virtual photon, $p$ is the initial four-momentum
of the target proton and $q'$ is the four-momentum of the produced
photon. The ``exclusive region'' was defined as $-$($1.5$\,GeV)$^{2} <
M_{\textrm{X}}^{2}$ $<$ (1.7\,GeV)$^{2}$, as in
ref.~\cite{Air09}. This exclusive region was shifted by up to
0.17\,GeV$^{2}$ for certain subsets of the data in order to reflect observed differences in the distributions of the electron and positron data samples~\cite{Bur10}. 
The data sample in the exclusive region contains events not only involving the production of real photons in which the proton remains intact, but also
events involving the excitation of the target proton to a $\Delta^+$
resonant state (``associated production''). The recoiling proton is not considered and the calorimeter resolution does not allow separation of all of the latter events from the rest of the data sample.
No systematic uncertainty is assigned for the contribution from these
events; they are treated as part of the signal. A Monte Carlo
calculation based on the parameterisation from ref.~\cite{Bra76} is
used to estimate the fractional contribution to the event sample from resonant
production in each kinematic bin; the uncertainty on this estimate
cannot be adequately quantified because no measurements have been made
in the H{\sc ermes} kinematic region. The results of the estimate, called the associated fractions and labelled ``Assoc. fraction'', are shown in the last row of figures~\ref{bsa_xbjrange}--\ref{bca_xbjrange2} in the results section. The method used to perform this estimation is described in detail in
ref.~\cite{Air08}.

\section{Experimental Extraction of Asymmetry Amplitudes}

The expectation value of the experimental yield $N$ is parameterised as
\begin{equation}
 \langle N(e_{\ell},P_{\ell},\phi)\rangle =
\mathcal{L}(e_{\ell},P_{\ell})\eta(e_{\ell},\phi)\sigma_{\textrm{UU}}
(\phi)
[1+P_{\ell}\mathcal{A}_{\textrm{LU}}^{\textrm{DVCS}}(\phi)+e_{\ell}P_{\ell}
\mathcal{A}_{\textrm{LU}}^{\textrm{I}}(\phi)+e_{\ell}\mathcal{A}_{\textrm{C}}
(\phi)],
\end{equation}
where $P_\ell$ is the beam polarisation, $\mathcal{L}$ is the integrated luminosity, $\eta$ is the detection
efficiency and d$\sigma_{\textrm{UU}}$ denotes the
cross section for an unpolarised target summed over both beam charges and
beam helicities. The asymmetries $\mathcal{A}_{\textrm{LU}}^{\textrm{DVCS}}(\phi)$, $\mathcal{A}_{\textrm{LU}}^{\textrm{I}}(\phi)$ and
$\mathcal{A}_{\textrm{C}}(\phi)$ are expanded in
$\phi$ as
\begin{equation}
 \mathcal{A}_{\textrm{LU}}^{\textrm{DVCS}}(\phi) \simeq 
A_{\textrm{LU,DVCS}}^{\sin\phi}\sin\phi 
+ \sum^{1}_{n=0} A_{\textrm{LU,DVCS}}^{\cos(n\phi)}\cos(n\phi), 
\label{aludvcs_asym}
\end{equation}
\begin{equation}
\mathcal{A}_{\textrm{LU}}^{\textrm{I}}(\phi) \simeq \sum^{2}_{n=1}
A_{\textrm{LU,I}}^{\sin(n\phi)}\sin(n\phi) 
+ \sum^{1}_{n=0} A_{\textrm{LU,I}}^{\cos(n\phi)}\cos(n\phi), 
\label{alui_asym}
\end{equation}
\begin{equation}
\mathcal{A}_{\textrm{C}}(\phi) \simeq \sum^{3}_{n=0}
A_{\textrm{C}}^{\cos(n\phi)}\cos(n\phi) 
+ A_{\textrm{C}}^{\sin\phi}\sin\phi,
\label{ac_asym}
\end{equation}
where the approximation is due to the truncation of the infinite
Fourier series that would describe exactly the fitted distribution. Only the $\sin(n\phi)$ terms of the
$\mathcal{A}_{\textrm{LU}}$ asymmetries and the $\cos(n\phi)$ terms of the
$\mathcal{A}_{\textrm{C}}$ asymmetry are motivated by the
  physical processes under investigation. The other terms
are included both as a consistency check for any off-phase
extraneous harmonics in the data and as a test of the
normalisation of the fit. These terms are expected to be
consistent with zero and are found to be so.

A maximum-likelihood fitting technique \cite{Bar90} was used to
extract the asymmetry amplitudes in each kinematic bin of $-t$, $x_{\textrm{B}}$ and $Q^{2}$.
This method, described in ref.~\cite{Air08}, fits the expected
azimuthal distribution function to the data without introducing binning effects in $\phi$.
Event weights are introduced in the fitting procedure to account for
luminosity imbalances with respect to the beam charge and polarisation.

The asymmetry amplitudes $A_{\textrm{LU,I/DVCS}}^{\sin(n\phi)}$ and
$A_{\textrm{C}}^{\cos(n\phi)}$ relate respectively to the Fourier
coefficients $s_{\textrm{unp},n}^{W}$ and $c_{\textrm{unp},n}^{I}$ from the interference and DVCS terms in eqs.~\ref{e:alui}-\ref{e:ac}. The asymmetry amplitudes
may also be affected by the lepton propagators and the other
$\phi$-dependent terms in the denominators in
eqs.~\ref{e:alui}-\ref{e:ac}.

The DVCS asymmetry amplitude $A^{\sin\phi}_{\textrm{LU,DVCS}}$ receives a
contribution from the $\mathcal{C}_{\textrm{unp}}^{\textrm{DVCS}}$-function,
which is bilinear in CFFs. However, this twist-3 amplitude is inherently small in HERMES kinematic conditions due to the size of the $s_{\textrm{unp},1}^{\textrm{DVCS}}$ Fourier coefficient compared to the contribution from the $c_{\textrm{unp},n}^{\textrm{BH}}$ coefficients in the denominator of eq.~\ref{e:aludvcs}. As a result of the more complicated dependence on the CFFs and this suppression, it
is more difficult to constrain GPDs via the measurement of $A^{\sin\phi}_{\textrm{LU,DVCS}}$ than from the kinematically-unsuppressed leading twist amplitudes.

The leading-twist asymmetry amplitudes are $A_{\textrm{C}}^{\cos(0\phi)}$, $A_{\textrm{C}}^{\cos\phi}$ and $A_{\textrm{LU,I}}^{\sin\phi}$, which are proportional to the Fourier coefficients $c_{\textrm{unp},0}^{\textrm{I}}$, $c_{\textrm{unp},1}^{\textrm{I}}$ and $s_{\textrm{unp},1}^{\textrm{I}}$ defined in eqs.~\ref{eq:c0}---\ref{eq:s1}. Whilst all of these amplitudes receive contributions from $\mathcal{C}_{\textrm{unp}}^{\textrm{I}}$, $c_{\textrm{unp},0}^{\textrm{I}}$ is kinematically suppressed in comparison to $c_{\textrm{unp},1}^{\textrm{I}}$, so $A_{\textrm{C}}^{\cos\phi}$ and $A_{\textrm{LU,I}}^{\sin\phi}$ are expected to have the largest magnitude in H{\sc ermes} kinematic conditions.

Although strictly dependent on higher-twist quantities, the asymmetry amplitudes $A_{\textrm{LU},\textrm{I}}^{\sin(2\phi)}$ and $A^{\cos(2\phi)}_{\textrm{C}}$ can also be expressed as having a dependence on $\mathcal{C}_{\textrm{unp}}^{\textrm{I}}$ using the Wandzura-Wilzcek approximation~\cite{Wan}, i.e. neglecting antiquark-gluon-quark contributions; these amplitudes that are dependent on higher-twist objects can therefore be considered as being functionally similar, but kinematically suppressed, when compared to the amplitudes that are dependent only on leading-twist objects.

The $A^{\cos(3\phi)}_{\textrm{C}}$ amplitude depends on the
$c_{\textrm{unp},3}^{\textrm{I}}$ Fourier coefficient and hence the
$\mathcal{C}_{\textrm{T,unp}}^{\textrm{I}}$-function. Although the CFFs
in this function are of leading twist, they relate to gluon helicity-flip GPDs and are thus suppressed by $\alpha_{\textrm{S}}/\pi$, where $\alpha_{\textrm{S}}$ is the strong coupling constant. Table~\ref{tab_amplitudes} presents the asymmetry amplitudes extracted in this analysis and, for each of them, the
related dominant Fourier coefficient and $\mathcal{C}$-function, and the twist-level at which the contributing GPDs enter.
\begin{table}
\begin{center}
\resizebox{\textwidth}{!} {
\begin{tabular}{|c|c|c|c|c|}
\hline
Asymmetry Amplitude& Fourier Coefficient& Dominant CFF Dependence & Twist-Level   \\ 
\hline
\hline
$A_{\textrm{LU,I}}^{\sin\phi}$ & $s_{\textrm{unp},1}^{\textrm{I}}$  &
$\mathfrak{Im}\mathcal{C}_{\textrm{unp}}^{\textrm{I}}$
&  2 \\
\hline
$A_{\textrm{LU,I}}^{\sin(2\phi)}$ & $s_{\textrm{unp},2}^{\textrm{I}}$ 
&
$\mathfrak{Im}\mathcal{C}_{\textrm{unp}}^{\textrm{I}}$
&  3 \\
\hline
\hline
$A_{\textrm{LU,DVCS}}^{\sin\phi}$ & $s_{\textrm{unp},1}^{\textrm{DVCS}}$ &
$\mathfrak{Im}\mathcal{C}_{\textrm{unp}}^{\textrm{DVCS}}$ &  3 \\
\hline
\hline
$A_{\textrm{C}}^{\cos(0\phi)}$ & $c_{\textrm{unp},0}^{\textrm{I}}$  &
$\mathfrak{Re}\mathcal{C}_{\textrm{unp}}^{\textrm{I}}$ & 2
\\
\hline
$A_{\textrm{C}}^{\cos\phi}$ & $c_{\textrm{unp},1}^{\textrm{I}}$  &
$\mathfrak{Re}\mathcal{C}_{\textrm{unp}}^{\textrm{I}}$ & 2
\\
\hline
$A_{\textrm{C}}^{\cos(2\phi)}$ & $c_{\textrm{unp},2}^{\textrm{I}}$ &
$\mathfrak{Re}\mathcal{C}_{\textrm{unp}}^{\textrm{I}}$ & 3 \\
\hline
$A_{\textrm{C}}^{\cos(3\phi)}$ & $c_{\textrm{unp},3}^{\textrm{I}}$ &
$\mathfrak{Re}\mathcal{C}_{\textrm{T,unp}}^{\textrm{I}}$ &  2 \\
\hline
 \end{tabular}
}
\caption{Asymmetry amplitudes that can
be extracted from the available data set, the related Fourier
coefficients, dominant $\mathcal{C}$-functions and twist-levels. {The constant K is defined in eq.~\ref{eq:K}}.}
\label{tab_amplitudes}
\end{center}
\end{table}
 
