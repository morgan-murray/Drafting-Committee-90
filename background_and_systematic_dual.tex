\section{Background Corrections and Systematic Uncertainties for the
  2006-2007 Data}
The extracted asymmetry amplitudes are subject to systematic uncertainties that
result from a combination of background processes, 
shifts in the missing-mass distributions and various detector and binning
effects determined in the same manner as in refs.~\cite{Air08,Air09}.

The contribution to the uncertainties on the amplitude measurements
arising from background in the data from neutral meson
production is \red{predominantly due to the failure to identify one of
  the two photons from the decay of produeced neutral mesons}. The procedure for estimating the corresponding
uncertainty on the measured amplitude values is described in detail in
refs.~\cite{Air08,Air09}. Each measurement in the $-t$, $x_{\textrm{B}}$ and
  $Q^{2}$ projections has this uncertainty estimated at the
  centre of the relevant kinematic bin and included as part of the
  total systematic uncertainty.

A contribution to the systematic uncertainties of the
  measured amplitudes arises from shifts in  the
missing-mass distributions. Such shifts appear in a comparison of electron and positron data~\cite{Zei09,Bur10}. One
quarter of the difference between the asymmetries extracted using the standard
and shifted missing-mass windows is taken as the corresponding systematic
uncertainty. 

The predominant contribution to the systematic uncertainty arises from detector
effects, including the acceptance of the spectrometer, smearing
effects due to detector resolution and radiative processes during
  detection, and
the finite bin width of the $-t$, $x_{\textrm{B}}$ and $Q^{2}$ projections. In order to quantify these effects, events were generated using a Monte Carlo simulation of the spectrometer that included them. An event
generator based on the GPD model described in ref.~\cite{Van99,Goe01} was
used for the simulation because it describes the data well and was employed in ref.~\cite{Air09}. Asymmetry amplitudes were extracted
from these simulated events using the same analysis procedure used to
extract amplitudes from experimental data. In each kinematic bin, the
systematic uncertainty was determined as the difference between the asymmetry amplitude reconstructed from the simulated data and that
calculated from the GPD model at the average $-t$, $x_{\textrm{B}}$ and
$Q^{2}$ value for that bin.

The total systematic uncertainty for the 2006-2007 data sample was
determined for each kinematic bin by adding in quadrature the
  uncertainties arising from the background correction, the
  missing-mass shifts, and the detector effects.  The 1996-2005 sample also has a
systematic uncertainty from misalignment of the
spectrometer~\cite{Air09}, which has been eliminated for the 2006-2007 data sample due to improved
  surveying measurements~\cite{Bur10}. The contribution from
each uncertainty to each amplitude extracted in a single\red{, overall} kinematic bin
from the 2006-2007 data  is summarised in Table
\ref{table_systematic_contributions_0607}.

The beam polarisation measurement has a total uncertainty of 2.8\% and 3.4\% for the 1996-2005 and 2006-2007 data taking periods respectively,
which is present in the beam-helicity amplitudes and is quoted as an
independent scale uncertainty without consideration in the other presented uncertainties.

The data sample in the exclusive region contains events involving both
production of real photons in which the proton remains intact and
events involving the excitation of the target proton to a $\Delta^+$
resonant state (``associated production''). The recoiling proton is not detected and
  the calorimeter resolution does not allow separation of the latter events from the rest of the data sample.
No systematic uncertainty is assigned for the contribution from these
events; they are treated as part of the signal. A Monte Carlo
calculation based on the parameterisation from ref.~\cite{Bra76} is
used to estimate the contribution to the event sample from resonant
production in each kinematic bin. The results, called the associated fractions and labelled ``Assoc. fraction'', are shown in the last row of each figure in the results section. The method used to perform this estimation is described in detail in
ref.~\cite{Air08}.

\begin{table}
 \begin{center}
\resizebox{\textwidth}{!} {
 \begin{tabular}{|c|c|c|c|c|c|c|}
  \hline
 & $A$ $\pm$ $\delta_{stat.}$ $\pm$ $\delta_{syst.}$ & Background & Missing-Mass Shift  & Detector Effects & & Total \\
  \hline
  \hline
  $A_{\textrm{LU,I}}^{\sin\phi}$ & -0.222  $\pm$  0.023  $\pm$   0.022 & 0.002 & 0.001 & 0.022 & & 0.022 \\
  \hline
  $A_{\textrm{LU,DVCS}}^{\sin\phi}$ & 0.005  $\pm$  0.023  $\pm$  0.003 & 0.002 & 0.002 & 0.001 & & 0.003 \\
  \hline
  $A_{\textrm{LU,I}}^{\sin(2\phi)}$ & 0.005  $\pm$  0.023  $\pm$   0.003 & 0.003 & 0.001 & 0.001 & & 0.003 \\
  \hline
  \hline
  $A_{\textrm{C}}^{\cos(0\phi)}$ & -0.024 $\pm$  0.004 $\pm$  0.011 & 0.001 & 0.003 & 0.010 & & 0.011 \\
  \hline
  $A_{\textrm{C}}^{\cos\phi}$ & 0.032  $\pm$  0.006 $\pm$   0.002 & 0.002 & 0.001 & 0.001 & & 0.002 \\
  \hline
  $A_{\textrm{C}}^{\cos(2\phi)}$ & -0.004  $\pm$  0.005  $\pm$   0.014 & 0.001 & 0.000 & 0.014 & & 0.014 \\
  \hline
  $A_{\textrm{C}}^{\cos(3\phi)}$ & 0.001  $\pm$   0.005   $\pm$   0.004 & 0.000 & 0.001 & 0.003 & & 0.004 \\
  \hline
 \end{tabular}
}
  \caption{The contributions to the \red{overall systematic 
uncertainties of the extracted asymmetry amplitudes} due to the
background correction, the time-dependent shifts of the missing-mass
distributions and detector effects for the 2006-2007 data. The total
systematic uncertainties \red{of the amplitudes}, shown in the
right-most column, are the individual contributions added in quadrature.}
  \label{table_systematic_contributions_0607}
\end{center}
\end{table}

